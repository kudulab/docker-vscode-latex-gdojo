\section{Application Presentation}

\subsection{The code origin}
\paragraph{}
The code used in this report was based on...

\subsection{A subsection with some Python}
\paragraph{}
The data was converted to json format by running:
\begin{lstlisting}
python text2json.py
\end{lstlisting}\

\paragraph{}
The main function is depicted below:

\begin{lstlisting}
def main():
    a = 42
    b = 882
    print(str(a+b))
\end{lstlisting}\

The result data files were put into $data/json$ directory. 

\section{A section with a list}
\paragraph{}
The results of this function were saved into three files under the $results/data$ directory:
\begin{itemize}
    \item file named after the parameters set, e.g. \begin{verbatim}pop_100-gen_100-cog_1.0-soc_3.0-speedmin_-3.0-speedmax_3.0.csv\end{verbatim}
    \item file that contained the mean values
    \item file that contained the mode values
\end{itemize}

\paragraph{}
A nice Python library to generate plots is: \href{https://matplotlib.org/index.html}{matplotlib}.


\section{A section with a figure}
\begin{figure}[H]
\includegraphics[width=180mm]{sections/app/{C101-pop_100-gen_400-cog_2.0-soc_2.0-speedmin_-3.0-speedmax_3.0}.png}
\end{figure}


\section{A section with a table}

\paragraph{}
\begin{tabularx}{0.8\textwidth} {
  | >{\centering\arraybackslash}X
  | >{\centering\arraybackslash}X
  | >{\centering\arraybackslash}X
  | >{\centering\arraybackslash}X
  | >{\centering\arraybackslash}X | }
 \hline
  & The 1st column & The 2nd column & The 3rd one & The last one \\
 \hline
 Row1  & 123 & 456 & 789 & 111 \\
 \hline
 Row2  & aaa & 456 & 789 & 111 \\
 \hline
 Row3  & 123 & 456 & 789 & 111 \\
 \hline
 Row4  & 123 & 456 & 789 & 111 \\
 \hline
\end{tabularx}
