\documentclass[12pt]{article}
\usepackage[utf8]{inputenc}
\usepackage[sfdefault]{carlito}
\usepackage{float}
\usepackage[a4paper,hmargin=1.5cm,vmargin=2cm]{geometry}
\usepackage{graphicx}
\usepackage{amsfonts}
\usepackage{textcomp}
\usepackage{hyperref}
\usepackage{listings}
\usepackage{array}
\usepackage{tabularx}
\usepackage{enumitem}
\setitemize{noitemsep,topsep=0pt,parsep=0pt,partopsep=0pt}
\setenumerate{noitemsep,topsep=0pt,parsep=0pt,partopsep=0pt}
\usepackage{wrapfig}
\usepackage[backend=bibtex]{biblatex}
\bibliography{bibliography.bib}

\graphicspath{{figures/}}

\title{\includegraphics[width=10mm]{Animals-Duck-icon.png}\vspace{5cm}\\
{\small The duck image comes from \href{http://www.iconarchive.com/show/windows-8-icons-by-icons8/Animals-Duck-icon.html}{www.iconarchive.com} } \\
\bf\textit{ The great title} \\
\vspace{4cm}}
\author{\begin{tabular}{cp{2cm}p{8cm}}
Anna Green  &&	anna@not-existent-domain.com\\
John Smith  &&	john@not-existent-domain.com\\
\end{tabular}\vspace{5cm}
}
\date{\today}


\begin{document}

\maketitle
\thispagestyle{empty}
\newpage
\thispagestyle{empty}
~\vspace{8cm}
\begin{abstract}
    %abstract
Lorem ipsum dolor sit amet, his amet libris virtute ad, eum labores invidunt ei. Pro option civibus sensibus te. Sea ne rebum equidem assentior. Ea melius viderer prodesset eum, oportere liberavisse definitiones in mea. In etiam utamur quo, ad docendi atomorum pro, sint feugiat duo id.

No vis quaestio sadipscing. Homero scripserit eu mel, his cu perfecto rationibus, diceret habemus eum an. Ad has modo electram. An iudico tollit probatus mel, no ius stet natum aperiam, in mel idque dicat.

Minimum intellegebat his no, nam cu mundi voluptaria. Debitis dolorem cum in, vis ut vero nullam molestiae. Mei an mucius singulis efficiantur, ne usu salutatus dissentias. Id brute fabellas iracundia eos, sea no eros invidunt assueverit. No assum lobortis sapientem has, cu omnis altera persequeris duo.

\end{abstract}
\newpage
\thispagestyle{empty}
\tableofcontents
\newpage
\section{Introduction} \label{intro}


\subsection{Introduction}
\paragraph{}
This is an introduction sentence. It is short. The whole section is short. But it should take more than one line so that we can see how the words are split across many lines.

\section{Application Presentation}

\subsection{The code origin}
\paragraph{}
The code used in this report was based on...

\subsection{A subsection with some Python}
\paragraph{}
The data was converted to json format by running:
\begin{lstlisting}
python text2json.py
\end{lstlisting}\

\paragraph{}
The main function is depicted below:

\begin{lstlisting}
def main():
    a = 42
    b = 882
    print(str(a+b))
\end{lstlisting}\

The result data files were put into $data/json$ directory. 

\section{A section with a list}
\paragraph{}
The results of this function were saved into three files under the $results/data$ directory:
\begin{itemize}
    \item file named after the parameters set, e.g. \begin{verbatim}pop_100-gen_100-cog_1.0-soc_3.0-speedmin_-3.0-speedmax_3.0.csv\end{verbatim}
    \item file that contained the mean values
    \item file that contained the mode values
\end{itemize}

\paragraph{}
A nice Python library to generate plots is: \href{https://matplotlib.org/index.html}{matplotlib}.


\section{A section with a figure}
\begin{figure}[H]
\includegraphics[width=180mm]{sections/app/{C101-pop_100-gen_400-cog_2.0-soc_2.0-speedmin_-3.0-speedmax_3.0}.png}
\end{figure}


\section{A section with a table}

\paragraph{}
\begin{tabularx}{0.8\textwidth} {
  | >{\centering\arraybackslash}X
  | >{\centering\arraybackslash}X
  | >{\centering\arraybackslash}X
  | >{\centering\arraybackslash}X
  | >{\centering\arraybackslash}X | }
 \hline
  & The 1st column & The 2nd column & The 3rd one & The last one \\
 \hline
 Row1  & 123 & 456 & 789 & 111 \\
 \hline
 Row2  & aaa & 456 & 789 & 111 \\
 \hline
 Row3  & 123 & 456 & 789 & 111 \\
 \hline
 Row4  & 123 & 456 & 789 & 111 \\
 \hline
\end{tabularx}

\newpage
\input{sections/wrapped-figure.tex}
\newpage
\section{Section to use biblatex}
This text comes from: \url{https://en.wikibooks.org/wiki/LaTeX/Bibliographies_with_biblatex_and_biber}

I doubt that there is any useful information here~\cite{wikibook}.

All we know is limited, apart from knowing the answer we all know. Or do we? Wombat and Koala have discovered some interesting things~\cite{wombat2016}.

Some people are too nosy. What can happen to them is described by Laura Lion~\cite[9]{lion2010}.
\addcontentsline{toc}{section}
         {\protect\numberline{References\hspace{-96pt}}}
\printbibliography
\end{document}
